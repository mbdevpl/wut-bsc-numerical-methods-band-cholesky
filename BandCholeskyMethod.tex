\documentclass{article}
\usepackage{fullpage}
\usepackage{algorithmic}
\usepackage{amssymb}
\usepackage{url}
\usepackage{color}

\providecommand{\e}[1]{\ensuremath{\times 10^{#1}}}

\begin{document}

\title{Band Cholesky Method implementation}
\date{April 18, 2011}
\author{Bysiek Mateusz, Witan Maciej (Computer Science A/R)}
\maketitle

\section{Task}

Design a MATLAB function that will solve a system of equations $Ax=B$ using the
band Cholesky method.

For each computed solution $\tilde{x}$ of the system $Ax=b$, form the relative
residual error:
\[ \frac{ \|b-A\tilde{x}\|_2 }{ \|A\|_F \|\tilde{x}\|_2 } \]

\section{Introdution to the method}

We have decided to design a modular function ``Band Cholesky Solver'', or $BCS$,
which will solve the problem iteratively, step by step. Divide and conquer
method allowed us to tackle the problem as a team, because on one hand both
team members had their own part of the work to do, but on the other hand we
then put the algorithms together, and synchronized them. The result is presented
below.

For this method, the input matrix $A$ must be symmetric positive definite, $A
\in \mathbb{R}^{n \times n}$. It also must be banded, with bandwith $w$. Matrix
$B$ should be of size $n \times m$.

\subsection{Fragmentation of the problem}

At first, the problem was $Ax = B$, where $B \in \mathbb{R}^{n \times m}$ had
$m$ columns: $b_1 \ldots b_m$. The problem was splitted to $m$ smaller
subproblems, such that: \[ \forall i \in \{1 \ldots m\} : Ax_i = b_i \]

Then, we did the following four steps for each column $b_i$, and received $m$
partial solutions.

\subsection{Cholesky decomposition}

Symmetric positive definite matrix $A \in \mathbb{R}^{n \times n}$, with bandwith $w$ is
given to the following algorithm.

\begin{algorithmic}
\STATE $L=A$
\FOR{$j = 1:n$}
	\FOR{$k = max(1,j-w):j-1$}
		\STATE $\lambda = min(k+w,n)$
		\STATE $L(j:\lambda,j) = L(j:\lambda,j) - L(j,k) L(j:\lambda,k)$
	\ENDFOR
	\STATE $\lambda = min(j+w,n)$
	\STATE $L(j:\lambda,j) = L(j:\lambda,j) / \sqrt{A(j,j)}$
\ENDFOR
\STATE $L = L - triu(L, 1)$
\end{algorithmic}

This algorithm ends with creation of matrix $L$, such that $A = LL^T$.
Initially, $L = A$, but L is overwritten step by step with correct values in the
lower-triangular area. After that, values over the diagonal are cleared.

\subsection{Solution of the preliminary system of equations}

Banded, lower triangular matrix $L \in \mathbb{R}^{n \times n}$ with bandwidth $w$ is
given to the algorithm below. Also, vector $b$ of length $n$ is given. 

\begin{algorithmic}
\STATE $y = b$
\FOR{$j = 1:n$}
	\FOR{$i = max(j, j-w):j-1$}
		\STATE $y(j) = y(j) - L(j,i) y(i)$
	\ENDFOR
	\STATE $y(j) = y(j)/L(j,j)$
\ENDFOR
\end{algorithmic}

Resulting vector $y \in \mathbb{R}^{n \times 1}$ is an intermediate solution
that is used in final calculations.

\subsection{Calculation of the final result}

The intermediate solution $y$ and upper-triangular matrix $U = L^T$ with
bandwith $w$ is given as the input.

\begin{algorithmic}
\STATE $x=y$
\FOR{$j = n:-1:1$}
	\STATE $x(j) = x(j)/U(j,j)$
	\FOR{$i = max(1,j-w):j-1$}
		\STATE $x(i) = x(i) - U(i,j) x(j)$
	\ENDFOR
\ENDFOR
\end{algorithmic}

\subsection{Calculating the error}

For every result, error $e$ is equal to $\|b-A\tilde{x}\|_2 /( \|A\|_F
\|\tilde{x}\|_2 )$.

\begin{algorithmic}
\STATE $e = norm(b-Ax,2) / (norm(A,"fro") norm(x,2))$
\end{algorithmic}

\section{Implementation}

The above-mentioned algorithms were implemented as separate functions.

\begin{enumerate}
	\item To split problem into smaller, we simply used $for$ loop.
	\item Cholesky decomposition: $cholband.m$
	\item Solution of the preliminary system of equations: $lsolve.m$
	\item Calculation of the final result: $usolve.m$
	\item Error estimation: $residerr.m$
\end{enumerate}

An aggregate function $BCS$ is used to launch those operations in sequence, as
well as to calculate residual error.

We also have constructed the following support files
\begin{itemize}
	\item solver that uses built-in functions, for comparison: $STS.m$
	\item menu to make using using our solution easier: $BCS\_menu.m$
	\item examples: $ex1.m$ $ex2.m$ $ex3.m$ 
\end{itemize}

\subsection{Examples}

For better presentation of how our solution works, we have included 3 examples
in our project.

\subsubsection{First example}

A is based on matrix of ones in the following way:

\begin{algorithmic}
\STATE $A = ones(8)$
\STATE $A = A-tril(A,-2)-triu(A,2)$
\STATE $A = A*A$
\STATE $A = A+eye(8)$
\end{algorithmic}

Vector b is obtained by $A \times ones(8, 1)$, and solving for $x$ using Band
Cholesky Solver gives exacly that vector.

\[ A = \left[
\begin{array}{cccccccc}
3 & 2 & 1 & 0 & 0 & 0 & 0 & 0 \\
2 & 4 & 2 & 1 & 0 & 0 & 0 & 0 \\
1 & 2 & 4 & 2 & 1 & 0 & 0 & 0 \\
0 & 1 & 2 & 4 & 2 & 1 & 0 & 0 \\
0 & 0 & 1 & 2 & 4 & 2 & 1 & 0 \\
0 & 0 & 0 & 1 & 2 & 4 & 2 & 1 \\
0 & 0 & 0 & 0 & 1 & 2 & 4 & 2 \\
0 & 0 & 0 & 0 & 0 & 1 & 2 & 3 \\
\end{array} \right], 
b = \left( \begin{array}{c} 
6 \\ 9 \\ 10 \\ 10 \\ 10 \\ 10 \\ 9 \\ 6
\end{array} \right), 
x = \left( \begin{array}{c} 
1\\ 1\\ 1\\ 1\\ 1\\ 1\\ 1\\ 1
\end{array} \right)
\]

Becasuse the relative error is very small, i.e $e =  6.8776\e{-16}$ indicating
that some deviations in results can appear not closer than at $16^{th}$ decimal
place this is thanks to the matrix A being well-conditioned as condtion number 
of matrix A is very close to the unity, namely: $cond(A) = 9.2909$.

\subsubsection{Second example}

A is based on product of Pascal and Hilbert matrices of size 11 in the
following way:

\begin{algorithmic}
\STATE $n = 11$
\STATE $A = hilb(n). * pascal(n)$
\STATE $A = A-tril(A,-2)-triu(A,2)$
\STATE $A = A^T * A$
\end{algorithmic}

Vector b is randolmy generated, but every time the same vector is obtained
thanks to $randn("state", 0)$ instruction:

\begin{algorithmic}
\STATE $randn("state", 0)$
\STATE $b = randn(n, 1)$
\end{algorithmic}

\[ b = \left(
\begin{array}{c}
  -1.2248\e{+00} \\
   7.6384\e{-01} \\
  -4.1902\e{-01} \\
   5.0462\e{-01} \\
   5.7743\e{-01} \\
   5.4095\e{-01} \\
  -6.0212\e{-01} \\
  -4.2490\e{-02} \\
   1.8388\e{-01} \\
  -6.5878\e{-02}
\end{array} \right)
\]

Solving for $x$ using Band Cholesky Solver the vector: 

\[ x_{BCS} = \left(
\begin{array}{c}
  -5.9584181126\underline{1340}\e{+00} \\
  +7.5253648228\underline{5158}\e{+00} \\
  -1.2786997060\underline{7469}\e{-01} \\
  -2.8406092064\underline{3483}\e{+00} \\
  +1.5221562124\underline{2016}\e{+00} \\
  +1.1955147494\underline{0896}\e{-01} \\
  -5.6776267273\underline{6740}\e{-01} \\
  +2.8906110676\underline{8170}\e{-01} \\
  +8.9938809722\underline{0923}\e{-03} \\
  -9.1118417264\underline{6874}\e{-02} \\
  +4.7840224822\underline{5803}\e{-02}
\end{array} \right)
\]

whereas the vector obtaied by using build-in Octave function $A \backslash b$:

\[ x_{OCT} = \left(
\begin{array}{c}
  -5.9584181126\underline{3058}\e{+00} \\
  +7.5253648228\underline{7429}\e{+00} \\
  -1.2786997060\underline{0663}\e{-01} \\
  -2.8406092064\underline{5384}\e{+00} \\
  +1.5221562124\underline{2985}\e{+00} \\
  +1.1955147494\underline{1742}\e{-01} \\
  -5.6776267274\underline{0454}\e{-01} \\
  +2.8906110677\underline{0057}\e{-01} \\
  +8.9938809722\underline{6832}\e{-03} \\
  -9.1118417265\underline{2829}\e{-02} \\
  +4.7840224822\underline{8929}\e{-02}
\end{array} \right)
\]
   
Becasuse the relative error is high, i.e $e =  3.59547536244005\e{-12}$
indicating that some deviations in results can appear even around $11^{th}$
decimal place this is due to the matrix A being ill-conditioned as condtion
number of matrix A is high $cond(A) = 1.08771091073209\e{9}$.

\subsubsection{Example no.3}

A is based on inverted magic square of size 9 in the following way:

\begin{algorithmic}
\STATE $n=9$
\STATE $A=magic(n)$
\STATE $A=inv(A)$
\STATE $A=(A-triu(A,3))-tril(A,-3)$
\STATE $A=A^T * A$
\end{algorithmic}

Vector $b$ is randolmy generated, but every time the same vector is obtained
thanks to $randn("state", 0)$ instruction:

\begin{algorithmic}
\STATE $randn("state", 0)$
\STATE $b = randn(n, 1)$
\end{algorithmic}

\[ b = \left( \begin{array}{c} 
  -1.2248\e{+00} \\
   7.6384\e{-01} \\
  -4.1902\e{-01} \\
   5.0462\e{-01} \\
   5.7743\e{-01} \\
   5.4095\e{-01} \\
  -6.0212\e{-01} \\
  -4.2490\e{-02} \\
   1.8388\e{-01}
\end{array} \right)
\]

Solving for x using Band Cholesky Solver, the vector: 

\[ x_{BCS} = \left(
\begin{array}{c}
  -8.54210613880\underline{332}\e{+06} \\
  -3.54982602561\underline{955}\e{+06} \\
  -2.32076201267\underline{229}\e{+06} \\
  -2.73885343436\underline{123}\e{+06} \\
  +2.91011321147\underline{994}\e{+07} \\
  -6.01576142237\underline{722}\e{+06} \\
  -6.06596003263\underline{988}\e{+06} \\
  -3.81886151192\underline{721}\e{+06} \\
  +1.37176843830\underline{206}\e{+07}
\end{array} \right)
\]

Whereas a vector obtaied by using build-in Octave function $A \backslash b$:

\[ x_{OCT} = \left(
\begin{array}{c}
%tez tylko 3 zaznaczyc (ww sensie w calym vektorze)
  -8.54210613880\underline{413}\e{+06} \\ 
  -3.54982602561\underline{894}\e{+06} \\
  -2.32076201267\underline{168}\e{+06} \\
  -2.73885343436\underline{059}\e{+06} \\
  +2.91011321147\underline{946}\e{+07} \\
  -6.01576142237\underline{623}\e{+06} \\
  -6.06596003263\underline{887}\e{+06} \\
  -3.81886151192\underline{656}\e{+06} \\
  +1.37176843830\underline{185}\e{+07}
\end{array} \right)
\]

Becasuse the relative error is quite significant, i.e $e =
1.62773786927727\e{-13}$ indicating that some deviations in results can appear 
even around $12^{th}$ decimal place this is due to the matrix A being 
ill-conditioned as condtion number of matrix A is quite high $cond(A) = 
6.17597814680354e+04$.

\subsection{How to use the function in Octave/MATLAB}

Usage of aggregate function $BCS$:

\[ [result\_matrix, error\_vector] = BCS(source\_matrix, parameter\_matrix,
bandwith\_value) \]

The script $BCS\_menu$ can be used for easier access to the function.

\section{Sources}

\begin{itemize}
	\item Algorithm 4.3.5, p. 155, G.H.Golub and Ch.F. Van Loan -
	\emph{Matrix Computations}, Third Edition, The Johns Hopkins University Press,
	Baltimore and London, 1996
	\item Article \emph{Forward Substitution}, retrieved: 7th
	April 2011, available at:
	\url{https://ece.uwaterloo.ca/~ece204/howtos/forward/} 
	\item Fortuna, Macukow, Wąsowski - \emph{Metody Numeryczne}
\end{itemize}
 
\end{document}
